\documentclass[a4paper,12pt]{jarticle}
\usepackage[dvipdfmx]{graphicx}
\usepackage{amsmath}
\usepackage{subfigure}
\usepackage{comment}

\setlength{\hoffset}{0cm}
\setlength{\oddsidemargin}{-3mm}
\setlength{\evensidemargin}{-3cm}
\setlength{\marginparsep}{0cm}
\setlength{\marginparwidth}{0cm}
\setlength{\textheight}{24.7cm}
\setlength{\textwidth}{17cm}
\setlength{\topmargin}{-45pt}

\renewcommand{\baselinestretch}{1.6}
\renewcommand{\floatpagefraction}{1}
\renewcommand{\topfraction}{1}
\renewcommand{\bottomfraction}{1}
\renewcommand{\textfraction}{0}
\renewcommand{\labelenumi}{(\arabic{enumi})}
%\renewcommand{\figurename}{Fig.} %図をFig.にする

\begin{comment}
%図のキャプションからコロン:を消す
\makeatletter
\long\def\@makecaption#1#2{% #1=図表番号、#2=キャプション本文
\sbox\@tempboxa{#1. #2}
\ifdim \wd\@tempboxa >\hsize
#1 #2\par 
\else
\hb@xt@\hsize{\hfil\box\@tempboxa\hfil}
\fi}
\makeatother
% 
\end{comment}

\begin{document}
%
<<<<<<< HEAD
\title{\vspace{-30mm}知能システム学特論レポート(第hdp2班)\\ 2016年7月14日(中間発表)}
=======
\title{\vspace{-30mm}知能システム学特論レポート(第hdp2班)\\ 2016年7月4日}
>>>>>>> 74e6d026618469359751756fba343a9ee2caccd4
\date{}
%
%
\maketitle
%
\vspace{-30mm}
%
%%%%%%%%%%%%%%%%%%%
\section{出席者}
%%%%%%%%%%%%%%%%%%
16344203 井上 聖也\\
~~~16344216 田中 良道\\
~~~16344217 津上 祐典\\
~~~16344229 沈 歩偉
%%%%%%%%%%%%%%%%%%%
<<<<<<< HEAD
\section{概要}
%%%%%%%%%%%%%%%%%%%

\subsection{Hadoopとは}

\subsection{Sparkとは}

\subsection{機械学習のテーマ}
学習テーマはスパムメールの分類とした.データセットとしてSpambase Data
Setを用いた.このデータセットは1813通のスパムメールと2788通の非スパムメー
ルから構成されており,すでに57次元のベクトルとして特徴量が抽出済みである.
学習アルゴリズムとして,ロジスティック回帰,ナイーブベイズを使用した.ロ
ジスティック回帰とは,識別関数としてシグモイド関数を用いた回帰モデルであ
る.パラメータを決定する際には確率的勾配法や最急降下法,準ニュートン法な
どが挙げられる.ナイーブベイズとは,ベイズの定理を用いた分類アルゴリズム
である.パラメータ推定には最尤法が用いられている.


\subsection{結果と考察}

%%%%%%%%%%%%%%%%%%
\section{今後の展望}
%%%%%%%%%%%%%%%%%
=======
\section{進捗状況}
%%%%%%%%%%%%%%%%%%%
\begin{itemize}
 \item スパムメールのデータセットに対してスタンドアローンモードで分類さ
	   せた.
 \item メンバーのPCでHadoop,Sparkの完全分散環境を構築した.
 \item Sparkの完全分散モードで機械学習のプログラムが実行できない.
 \item 分類後の評価の方法について調査した.
\end{itemize}

%%%%%%%%%%%%%%%%%%
\section{その他}
%%%%%%%%%%%%%%%%%
\begin{itemize}
 \item 今後,Hadoop,Sparkを組み合わせて完全分散処理を行う.
 %\item 今後,制作したコードで完全分散処理させる.
 \item 学習パラメータを調整してみる.
 \item 特徴量が抽出されていないデータに対しても分類してみる.
\end{itemize}
>>>>>>> 74e6d026618469359751756fba343a9ee2caccd4

\end{document}

\documentclass[a4paper,12pt]{jarticle}
\usepackage[dvipdfmx]{graphicx}
\usepackage{amsmath}
\usepackage{subfigure}
\usepackage{comment}

\setlength{\hoffset}{0cm}
\setlength{\oddsidemargin}{-3mm}
\setlength{\evensidemargin}{-3cm}
\setlength{\marginparsep}{0cm}
\setlength{\marginparwidth}{0cm}
\setlength{\textheight}{24.7cm}
\setlength{\textwidth}{17cm}
\setlength{\topmargin}{-45pt}

\renewcommand{\baselinestretch}{1.6}
\renewcommand{\floatpagefraction}{1}
\renewcommand{\topfraction}{1}
\renewcommand{\bottomfraction}{1}
\renewcommand{\textfraction}{0}
\renewcommand{\labelenumi}{(\arabic{enumi})}
%\renewcommand{\figurename}{Fig.} %図をFig.にする

\begin{comment}
%図のキャプションからコロン:を消す
\makeatletter
\long\def\@makecaption#1#2{% #1=図表番号、#2=キャプション本文
\sbox\@tempboxa{#1. #2}
\ifdim \wd\@tempboxa >\hsize
#1 #2\par 
\else
\hb@xt@\hsize{\hfil\box\@tempboxa\hfil}
\fi}
\makeatother
% 
\end{comment}

\begin{document}
%
\title{\vspace{-30mm}知能システム学特論レポート(第hdp2班)\\ 2016年6月20日}
\date{}
%
%
\maketitle
%
\vspace{-30mm}
%
%%%%%%%%%%%%%%%%%%%
\section{出席者}
%%%%%%%%%%%%%%%%%%
16344203 井上 聖也\\
~~~16344216 田中良道\\
~~~16344217 津上 祐典\\
~~~16344229 沈 歩偉
%%%%%%%%%%%%%%%%%%%
\section{進捗状況}
%%%%%%%%%%%%%%%%%%%
\begin{itemize}
 \item Hadoopを使って機械学習を行うためのソフトウェアを調査した.
	   \begin{itemize}
		\item 具体的にはMahoutやSparkなどのソフトウェアがあることがわかった.
		\item Sparkの使用を検討している.
		\item Sparkをインストールした. 
	   \end{itemize} 
 \item HadoopとSparkを組み合わせる方法を調査した.
\end{itemize}

%%%%%%%%%%%%%%%%%%
\section{その他}
%%%%%%%%%%%%%%%%%
\begin{itemize}
 \item Sparkのサンプルプログラムは動いた.
 \item 学習テーマ・アルゴリズムについては再検討中である.
\end{itemize}

\end{document}

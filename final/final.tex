\documentclass[a4paper,12pt]{jarticle}
\usepackage[dvipdfmx]{graphicx}
\usepackage{amsmath}
\usepackage{subfigure}
\usepackage{comment}

\setlength{\hoffset}{0cm}
\setlength{\oddsidemargin}{-3mm}
\setlength{\evensidemargin}{-3cm}
\setlength{\marginparsep}{0cm}
\setlength{\marginparwidth}{0cm}
\setlength{\textheight}{24.7cm}
\setlength{\textwidth}{17cm}
\setlength{\topmargin}{-45pt}

\renewcommand{\baselinestretch}{1.6}
\renewcommand{\floatpagefraction}{1}
\renewcommand{\topfraction}{1}
\renewcommand{\bottomfraction}{1}
\renewcommand{\textfraction}{0}
\renewcommand{\labelenumi}{(\arabic{enumi})}
%\renewcommand{\figurename}{Fig.} %図をFig.にする

\begin{comment}
%図のキャプションからコロン:を消す
\makeatletter
\long\def\@makecaption#1#2{% #1=図表番号、#2=キャプション本文
\sbox\@tempboxa{#1. #2}
\ifdim \wd\@tempboxa >\hsize
#1 #2\par 
\else
\hb@xt@\hsize{\hfil\box\@tempboxa\hfil}
\fi}
\makeatother
% 
\end{comment}

\begin{document}
%
\title{\vspace{-30mm}知能システム学特論 \ 最終レポート\\ Hdp第2班 16344217 津上祐典\\ 2016年7月14日}
\date{}
%
%
\maketitle
%
\vspace{-30mm}
%
%%%%%%%%%%%%%%%%%%%
 \section{テーマ}
%%%%%%%%%%%%%%%%%%
機械学習の分散並列処理
%%%%%%%%%%%%%%%%%%%
\section{概要}
%%%%%%%%%%%%%%%%%%%

%%%%%%%%%%%%%%%%%%%%%%%
\subsection{Hadoopとは}
%%%%%%%%%%%%%%%%%%%%%%%
Hadoopとはビッグデータを複数のPCで分散処理を可能にするフレームワークであ
る.ー台マスターサーバとその配下にある複数のスレーブサーバのによって分散
処理を実現している.Hadoopは分散ファイルシステム(Hadoop Distributed File
System),並列分散処理フレームワーク(MapReduce
Framework)より構成されている.分散ファイルシステムとは複数のスレーブサーバを一つのストレー
ジとして扱うファイルシステムである.身近な例で言うとクラウドやネットワー
クHDD(NAS)などが挙げられる.並列分散処理フレームワークでは与えられたデー
タから欲しいデータの抽出と分解するMAP処理,それらのデータを集計する
Reduce処理が行われる.MapReduce処理を複数のスレーブサーバで行うことで分
散処理を可能にし,ビッグデータを効率よく扱うことができる.

%%%%%%%%%%%%%%%%%%%%%%
\subsection{Sparkとは}
%%%%%%%%%%%%%%%%%%%%%%
Apache SparkはUC BerkeleyのAMPLabにて開発された大規模データの分散処理フ
レームワーク.RDDの導入によって,計算速度がHadoopより大きく上がっていた.
特に再帰など頻繁的にデータを読むと書くのアルゴリズム,性能が100倍上がれ
ると言われている.しかし性能の代わり,メモリの消耗もHadoopより遥に激しくなる.
Spark自身が分散ファイルズシステムを持っていないため,他の分散処理ファイ
ルズシステムと連携を取らなければいけない.HadoopのHDFSはその中でよく使っ
ているシステムの一つ.

%%%%%%%%%%%%%%%%%%%%
\subsection{機械学習}
%%%%%%%%%%%%%%%%%%%%

%%%%%%%%%%%%%%%%%%%%%%%%%%%%%%%%%%%%%%
\subsubsection{データセットとアルゴリズム}
%%%%%%%%%%%%%%%%%%%%%%%%%%%%%%%%%%%%%%
学習テーマはスパムメールの分類とした.データセットとしてSpambase Data
Setを用いた.このデータセットは1813通のスパムメールと2788通の非スパムメー
ルから構成されており,すでに57次元のベクトルとして特徴量が抽出済みである.
学習アルゴリズムとして,ロジスティック回帰,ナイーブベイズ,SVMを使用した.ロ
ジスティック回帰とは,識別関数としてシグモイド関数を用いた回帰モデルであ
る.正しく分類される確率を最大化(学習)することがこのアルゴリズムの目的
である.パラメータを決定する際には確率的勾配法や最急降下法,準ニュートン法な
どが挙げられる.ナイーブベイズとは,ベイズの定理を用いた分類アルゴリズム
であり,各クラスに分類される確率を学習し,最も確率の高いクラスを出力する.
SVMとは,訓練データのクラス同士で
一番近いサンプル(サポートベクトル)と分離超平面との距離が最大になるように
学習を行う線形分離器である.

%%%%%%%%%%%%%%%%%%%%%%
\subsubsection{評価法}
%%%%%%%%%%%%%%%%%%%%%%
学習精度の評価に際して,今回はホールドアウト法を採用しデータセットを$7:3$の割合で訓練デー
タとテストデータに分けた.ホールドアウト法とはデータをある割合で訓練データ,テスト
データに分割し,学習結果を評価する一手法である.他の評価方法として,k-分
割交差検証(k-fold Cross Validation)やLeave-One-Out交差検証(LOOCV)などが
ある.また,訓練デー
タで学習したモデルでテストデータの分類を実行し,そのときの再現率,適合率お
よびArea Under P-R curveを算出して評価値とした.メール分類における再現率と
は,正解数に対するスパム(もしくは非スパ
ム)と正しく判定できたものの割合である.また適合率とは,スパム(もしくは非
スパム)だと分類して本当にスパム(もしくは非スパム)だったものの割合である.
スパムメールの分類においては,スパムに対する適合率が低いと非スパムメール
を誤って読み過ごすことになるので,これを高く保った上でその他の評価値もな
るべく高くするのが良い.また,再現率と適合率は一般的にトレードオフの関係
にある.Area Under P-R curveとは,陰性・陽性の判定閾値を変動させたときに再現率と適合率の関係を描写したものであるPR曲線の図における,曲線よりも下の
面積を全体の面積の比率で表現した値であり,1に近いほど高い分類性能であるこ
とを示している.
%%%%%%%%%%%%%%%%%
\subsection{実験}
%%%%%%%%%%%%%%%%%
3つのアルゴリズムを用いてメールの分類を行った.その結果を表\ref{tab:実験
結果}に示す.また,ロジスティック回帰モデルで設定パラメータを実
験的にチューニングして求めた際の結果を表\ref{tab:パラメータのチューニング前後の比較}に示
す.

\begin{table}[bt]
\centering
\caption{実験結果}
\label{tab:実験結果}
\fontsize{9pt}{10pt}\selectfont
\begin{tabular}{|c|c|c|c|c|c|} \hline
 &非スパム再現率&スパム再現率&非スパム適合率&スパム適合率&AUC(PR) \\ \hline
SVM& 76.13\%& 85.79\%& 88.90\%& 70.61\% & 0.8105 \\ \hline
ロジスティック回帰&91.92\% & 90.41\% & 93.48\% & 88.21\% & 0.9123 \\ \hline
ナイーブベイズ& 83.85\% & 66.25\% & 78.79\%  & 73.28\% & 0.7652  \\ \hline
\end{tabular}
\end{table}

\begin{table}[bt]
\centering
\caption{パラメータのチューニング前後の比較}
\label{tab:パラメータのチューニング前後の比較}
\fontsize{9pt}{10pt}\selectfont
\begin{tabular}{|c|c|c|} \hline
 &チューニング前&チューニング後 \\ \hline
正則化関数& L2ノルム & L1ノルム \\ \hline
正則化係数&0.01 & 0.002 \\ \hline
学習繰り返し回数  &10 & 25  \\ \hline \hline
非スパム再現率& 88.60\%& 91.92\% \\ \hline
スパム再現率& 88.81\%& 90.41\% \\ \hline
非スパム適合率& 92.21\%& 93.48\% \\ \hline
スパム適合率& 83.89\%& 88.21\% \\ \hline
ACU(PR)& 0.8859 &0.9123 \\ \hline
\end{tabular}
\end{table}

%%%%%%%%%%%%%%%%%%%
\section{自分の分担範囲}
%%%%%%%%%%%%%%%%%%

%%%%%%%%%%%%%%%%%%%
\section{感想}
%%%%%%%%%%%%%%%%%%

%%%%%%%%%%%%%%%%%%%
\section{評価}
%%%%%%%%%%%%%%%%%%
16344203 井上 聖也\\
~~~16344216 田中 良道\\
~~~16344217 津上 祐典\\
~~~16344229 沈 歩偉

%参考文献
\begin{thebibliography}{99}
\addcontentsline{toc}{section}{参考文献}

 \bibitem{jamming} 杉山将,
		 "イラストで学ぶ機械学習",講談社, 2013.

\bibitem{jamming} 下田倫大ら,
		"詳解ApacheSpark",技術評論社, 2016.

 \bibitem{jamming} 海野裕也ら,
		 "オンライン機械学習",講談社, 2015.

\end{thebibliography}

\end{document}
